\section{Proposed Work}\label{sec:proposed_work}

\subsection{Problem Statement}\label{subsec:problem_Statement}

Accurate discrimination of soft‐tissue types in X‐ray computed tomography (CT) remains a
fundamental challenge in medical imaging. Conventional single‐energy CT produces grayscale
images in which different materials with similar attenuation coefficients (e.g.\,muscle 
vs.\,iodine‐enhanced blood or bone) can appear indistinguishable, leading to diagnostic 
ambiguity. Dual‐energy and photon‐counting CT systems acquire multiple energy‐resolved 
measurements, but extracting robust tissue‐specific maps from these spectral data is 
nontrivial: standard material‐decomposition methods are sensitive to noise, beam‐hardening, 
and detector imperfections, and purely data‐driven deep‐learning approaches often fail to 
generalize beyond their training domain.

This project proposes to address these limitations by developing a \emph{physics‐informed neural network} 
(PINN) that directly incorporates the known Beer–Lambert attenuation law, $I = I_0 e^{-\mu x}$, and the
two dominant interaction mechanisms—photoelectric absorption and Compton scattering—into its architecture 
and training loss. By decomposing each pixel’s dual‐energy attenuation pair \([\mu_{\rm low},\,
\mu_{\rm high}]\) into physically meaningful photoelectric and Compton components and enforcing 
consistency with both the measured attenuation maps and sinogram data, this approach aims to (1)
improve classification accuracy of key tissue types (adipose, fibroglandular, calcification) and (2) 
enhance robustness to noise and out‐of‐distribution scenarios. This integration of first‐principles 
physics with modern deep learning may achieve more reliable, interpretable, and generalizable 
spectral CT tissue characterization.

\subsection{Data Preparation}\label{sec:data_preparation}

The dataset from AAPM \cite{AAPM2024SpectralCT} contains data in compressed numpy array format. Both
`lowkVpTransmission` and `highkVpTransmission` datasets contain the two energy beams, 50 kVp and 80
kVp, respectively. The transmission data contains the normalized spectrum of the number of photons
detected after passing through tissue. Ground trut tissue maps are also provided for the simulated
Adipose, Fibrogandular, and Calcification tissues.

To prepare the dataset, an \emph{MLPTestTraindataset} class is used to:

\begin{itemize}
    \item load the raw data,
    \item compute the attenuation coefficients,
    \item vectorize the attenuation coefficients,
    \item create labels from adipose, fibroglandular, and calcification ground truth data,
    \item flatten to (num\_pixels,)
    \item apply scaling to the attenuation coefficients
    \item compute the prototypes
    \item split the data into train (80\%) and test (80\%) sets.
\end{itemize}

The intent of this project is to calculate the attenuation using the from the transmission data and stack the results
into 2-component feature vectors $[\mu_{\text{low}}, \mu_{\text{high}}]$.
translate the data into a vectorized low and high energy spectrum sets


\subsection{Methodology}\label{subsec:price_prediction_methodology}

