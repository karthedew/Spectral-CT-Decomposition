\section{Proposed Work}\label{sec:proposed_work}

\subsection{Problem Statement}\label{subsec:problem_Statement}

Accurate discrimination of soft‐tissue types in X‐ray computed tomography (CT) remains a
fundamental challenge in medical imaging. Conventional single‐energy CT produces grayscale
images in which different materials with similar attenuation coefficients (e.g.\,muscle 
vs.\,iodine‐enhanced blood or bone) can appear indistinguishable, leading to diagnostic 
ambiguity. Dual‐energy and photon‐counting CT systems acquire multiple energy‐resolved 
measurements, but extracting robust tissue‐specific maps from these spectral data is 
nontrivial: standard material‐decomposition methods are sensitive to noise, beam‐hardening, 
and detector imperfections, and purely data‐driven deep‐learning approaches often fail to 
generalize beyond their training domain.

This project proposes to address these limitations by developing a \emph{physics‐informed neural network} 
(PINN) that directly incorporates the known Beer–Lambert attenuation law and the two dominant 
interaction mechanisms—photoelectric absorption and Compton scattering—into its architecture 
and training loss. By decomposing each pixel’s dual‐energy attenuation pair \([\mu_{\rm low},\,
\mu_{\rm high}]\) into physically meaningful photoelectric and Compton components and enforcing 
consistency with both the measured attenuation maps and sinogram data, our approach aims to (1)
improve classification accuracy of key tissue types (adipose, fibroglandular, bone) and (2) 
enhance robustness to noise and out‐of‐distribution scenarios. This integration of first‐principles 
physics with modern deep learning promises more reliable, interpretable, and generalizable 
spectral CT tissue characterization.

\subsection{Data Preparation}\label{sec:data_preparation}

\subsection{Methodology}\label{subsec:price_prediction_methodology}

