\section{Conclusion}\label{sec:conclusion}

This work introduces a deep learning framework for spectral CT tissue decomposition, leveraging U-Net 
architectures to segment and quantify adipose, fibroglandular, and calcified tissues from dual-energy attenuation inputs. 
From utilizing the Beer–Lambert law, domain-specific feature construction, and deep convolutional 
networks, the method achieves high segmentation fidelity across tissue types, including rare calcification patterns.

Quantitative evaluation shows that the deeper \texttt{UNet512} model achieves the best performance, with a final validation 
loss of 0.0292 and mean absolute error below 0.003 for all tissue categories. Predicted tissue compositions closely match 
ground truth distributions, highlighting the model`s clinical relevance.

Overall, this study demonstrates that coupling physics-based domain knowledge with deep learning architectures offers a 
powerful and interpretable approach to spectral CT analysis. Future work will extend this model to real-world datasets, 
investigate 3D volumetric extensions, and incorporate uncertainty estimation to further enhance clinical utility.
