\begin{abstract}

Spectral computed tomography (CT) enables material-specific imaging by exploiting energy-dependent x-ray attenuation. 
However, accurate tissue decomposition—especially of low-prevalence classes like calcifications—remains challenging 
due to class imbalance, noise sensitivity, and limited ground truth data. This study introduces a physics-informed 
deep learning framework for tissue segmentation in dual-energy spectral CT. By leveraging the Beer–Lambert attenuation 
law and incorporating dual-energy attenuation maps \([\mu_{\text{low}}, \mu_{\text{high}}]\) as inputs, we train 
U-Net-based convolutional neural networks (CNNs) to segment adipose, fibroglandular, and calcified tissue. Two 
architectures, \texttt{UNet256} and \texttt{UNet512}, are evaluated, with the deeper \texttt{UNet512} model achieving 
superior performance (validation loss = 0.0292). Quantitative analysis shows close agreement between predicted and 
true tissue compositions, including for rare calcifications. This work demonstrates that embedding first-principles 
physics into CNN design enhances interpretability and segmentation accuracy, offering a reliable approach for tissue 
decomposition in spectral CT.

\vskip 2mm

  \textbf{Keywords:} Physics-Informed, Duel-Energy CT, Clasification, Machine Learning

\end{abstract}
