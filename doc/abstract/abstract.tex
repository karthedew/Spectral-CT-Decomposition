\begin{abstract}

Spectral computed tomography (Spectral CT) enables material-specific imaging by exploiting energy-dependent x-ray 
attenuation. However, accurate tissue decomposition—especially of low-prevalence classes like calcifications—remains 
challenging due to class imbalance, noise sensitivity, and limited ground truth data. This study introduces a
deep learning framework for tissue segmentation in dual-energy spectral CT. By leveraging the Beer–Lambert attenuation 
law and incorporating dual-energy attenuation maps \([\mu_{\text{low}}, \mu_{\text{high}}]\) as inputs, a U-Net-based 
convolutional neural networks (CNNs) is trained to segment adipose, fibroglandular, and calcified tissue. Two 
architectures, \texttt{UNet256} and \texttt{UNet512}, are evaluated, with the deeper \texttt{UNet512} model achieving 
superior performance (validation loss = 0.0292). Quantitative analysis shows close agreement between predicted and 
true tissue compositions, including for rare calcifications. This work demonstrates that using Beer–Lambert's law 
with Filtered Back Projection into CNN design enhances interpretability and segmentation accuracy, offering a reliable 
approach for tissue decomposition in spectral CT.

\vskip 2mm

  \textbf{Keywords:} Spectral CT, Clasification, Machine Learning, UNet, CNN

\end{abstract}
