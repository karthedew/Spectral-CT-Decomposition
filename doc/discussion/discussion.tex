\section{Discussion}\label{sec:discussion}

The results demonstrate the feasibility and effectiveness of using UNet convolutional neural networks for 
tissue segmentation in spectral CT imaging. By explicitly encoding dual-energy attenuation inputs and aligning the model 
output with the physical characteristics of adipose, fibroglandular, and calcified tissues, the network is able to 
generalize well across heterogeneous tissue distributions.

A key observation is the improvement in performance with increased network depth, as seen in the comparison between 
\texttt{UNet256} and \texttt{UNet512}. The deeper architecture with an additional downsampling path captures more 
complex spatial and spectral correlations, leading to a lower validation loss and improved tissue classification 
accuracy. This is particularly evident in the accurate prediction of rare calcification features, despite significant 
class imbalance in the training data.

The use of domain-specific input features—namely, dual-energy attenuation values computed from raw transmission 
data—proved essential. These features directly encode physical interaction mechanisms (photoelectric and Compton 
effects), helping the model learn more meaningful and generalizable patterns compared to raw image pixels or naive 
inputs.

There are several considerations that should be considered in future work. Firstly, the training dataset was synthetically 
generated, and real-world generalization to clinical CT scans may require transfer learning or fine-tuning on real data. 
Secondly, the class imbalance, while mitigated to some extent by network design, may benefit from more sophisticated loss 
functions or targeted sampling strategies. Finally, although model convergence occurred rapidly (within five epochs), longer 
training and ensemble methods could further improve robustness.

